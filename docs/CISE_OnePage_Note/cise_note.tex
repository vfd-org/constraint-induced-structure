\documentclass[11pt,letterpaper]{article}
\usepackage[margin=0.75in]{geometry}
\usepackage{graphicx}
\usepackage{amsmath}
\usepackage{hyperref}
\usepackage{enumitem}
\usepackage{booktabs}

\setlength{\parindent}{0pt}
\setlength{\parskip}{6pt}

\title{\textbf{CISE: Constraint-Induced Structure Explorer}\\[0.3em]
\large A Computational Experiment on Constraint-Induced Structure}
\author{Lee Smart\\
\small Vibrational Field Dynamics Institute\\
\small \texttt{contact@vibrationalfielddynamics.org}}
\date{}

\begin{document}

\maketitle
\vspace{-1em}

\hrule
\vspace{0.5em}

\section*{What It Is}

CISE is a computational framework that demonstrates how simple coherence constraints induce non-trivial structure in random ensembles. Given baseline samples from standard distributions (Gaussian, Uniform), we apply constraints via energy-based reweighting:
\[
w(z) = \exp(-\beta \cdot E(z))
\]
where $E(z)$ is a penalty function (smoothness, L1, low-rank, hierarchy) and $\beta$ controls constraint strength.

\textbf{Core finding:} Constraints induce measurable structure---dimensional concentration, hierarchical decay, and clustering---that persists even after controlling for norm shrinkage.

\section*{What It Is NOT}

\begin{itemize}[nosep,leftmargin=1.5em]
    \item Not a physics model or simulation
    \item Not a claim about physical reality
    \item Not a predictor of experimental outcomes
    \item Not an implementation of any physical theory
    \item Does not require GPU, internet, or external APIs
\end{itemize}

This is purely a \textbf{constraint experiment} studying how penalty functions shape distributions.

\section*{Key Results}

\vspace{-0.5em}
\begin{center}
\small
\begin{tabular}{lll}
\toprule
\textbf{Metric} & \textbf{Change} & \textbf{Interpretation} \\
\midrule
Norm & Decreases & Constraints favor smaller magnitudes \\
Gini coefficient & Increases & Values become more unequal \\
PCA dimensions (90\%) & Decreases & Dimensional concentration \\
Rank proxy (matrices) & Decreases & Effective low-rank structure \\
ESS ratio & Decreases with $\beta$ & Measure concentration \\
\bottomrule
\end{tabular}
\end{center}

\section*{Key Figures}

\textit{(Insert from \texttt{outputs/release\_run/figures\_release/})}

\begin{enumerate}[nosep,leftmargin=1.5em]
    \item \textbf{ESS vs $\beta$}: Effective sample size drops as constraint strength increases
    \item \textbf{Norm distribution}: Constrained samples shift to lower norms
    \item \textbf{PCA variance}: Variance concentrates in fewer dimensions
    \item \textbf{SV spectrum}: Low-rank constraint suppresses smaller singular values
\end{enumerate}

\section*{Anti-Dismissal Control}

\textbf{Critique:} ``Isn't this just norm shrinkage?''

\textbf{Control:} We construct a norm-matched baseline whose norm distribution matches the constrained distribution. Structure differences persist---the participation ratio and Gini deltas remain non-zero after norm matching.

This demonstrates that constraint \textit{geometry}, not just magnitude reduction, drives structural changes.

\section*{Non-Claims}

\begin{itemize}[nosep,leftmargin=1.5em]
    \item Does not claim to explain any physical phenomenon
    \item Does not predict experimental outcomes
    \item Does not propose modifications to physical theories
    \item Does not claim universality beyond the tested ensembles
    \item Results are distributional observations, not physical laws
\end{itemize}

\section*{Reproducibility}

Deterministic with fixed seed (1337). Runs in $<$2 minutes on CPU.

\begin{verbatim}
pip install -e .
python scripts/run_release.py
\end{verbatim}

\vspace{0.5em}
\hrule
\vspace{0.5em}

\textbf{Contact:} Lee Smart, Vibrational Field Dynamics Institute --- \texttt{contact@vibrationalfielddynamics.org}

\textbf{Repository:} \url{https://github.com/vfd-org/constraint-induced-structure}

\textbf{License:} MIT

\vspace{0.3em}
\textit{This is a constraint experiment, not a physics claim.}

\end{document}
